%!TEX root = ../main.tex
\section{Stack Machine}
\label{sec:stack-machine}

% - Hvad er en stack machine?
% - Udfordringer
% - Løsninger
% - Program representation
% - Compilation
% - Execution

Now that we have a well-typed representation, we would like to be able to compile and execute it. This means we need a compiler to translate the well-typed representation to instructions, and a machine to run these instructions. Due to its simplicity, we have chosen to implement a stack machine.

A minimal stack machine needs only a few components: a list of instructions, a program counter (PC), a stack, and a stack pointer (SP). The program counter records how far into the array of instructions the program is, and the stack pointer points to the topmost element of the stack. As each instruction is executed, the stack is manipulated and the program counter is increased. Because each instruction operates on the top of the stack, programs are written in a postfix style. For example, if you want to add the constants 3 and 4 together, the instructions required would be: [PUSH 3, PUSH 4, ADD]. The first two instructions push the constants onto the stack, and the third instruction adds them together, and leaves the result on the stack. Thus the stack effect of the PUSH instructions is to add one value to the stack each, and the effect of the ADD instruction is to consume two values from the stack, and leave one value. The stack effect of the two instruction can be written (0, 1) and (2, 1), respectively.

\subsubsection{Problems}
In most stack machine implementations, many errors can occur at run time. For example, running the program [PUSH 3, ADD] with an empty stack will result in a stack underflow error, as the ADD instruction is expecting 2 values on the stack\todo{More errors?}. It is up to the compiler to make sure all the instructions are correct. Unfortunately, most compilers are not accompanied by a proof of correctness. Instead they are thoroughly tested. As a consequence, even if you have proved the correctness of your program, as soon as you compile it with an unverified compiler, you lose all assurances, as Xavier Leroy points out in an article on the verified C compiler CompCert\,\cite{Leroy_formalverification}.

\subsection{Desired Qualities}
While it is beyond the scope of this project to produce a fully verified compiler and stack machine, we can use dependent types to make certain guarantees about the compiler, the programs it produces, and the stack machine that runs the programs. Specifically, we would like a program representation that encodes the stack effect of the instructions that comprise it, and a stack machine that only accepts such a \textit{well-stacked} program that will run with an empty stack, without any stack underflow errors. The stack machine will be very simple. Instead of using a program counter and a stack pointer, the program and the stack themselves will be passed as parameters to the stack machine. As the stack machine is only used to run programs compiled from the well-typed representation, all values will be placed on the stack. This means a heap is not needed, and was not implemented.

%!TEX root = ../main.tex
\subsection{Programs}
\label{sec:program}
As previously mentioned a program run by a stack machine is a list of instructions. Our machine's instruction set can be seen in Figure~\ref{fig:inst_set}. Most instructions should be self-explanatory, but for further explanation our stack machine is inspired by Peter Sestofts abstract machine for micro-C\,\cite[pp. 157-161]{Sestoft:PLC}. When executed, an instructions has an effect on the stack specific to each instruction.

\begin{figure}
\centering{
\begin{tabularx}{\textwidth}{ | X X X | }
  \hline
  \textbf{Instruction} & \textbf{Stack Before} & \textbf{Stack After} \\ \hline
  PUSH $i$ & $s$ & $s$, $i$ \\ 
  ADD & $s$, $i_{1}$, $i_{2}$ & $s$, $(i_{1}+i_{2})$ \\
  SUB & $s$, $i_{1}$, $i_{2}$ & $s$, $(i_{1}+i_{2})$ \\
  MUL & $s$, $i_{1}$, $i_{2}$ & $s$, $(i_{1}+i_{2})$ \\
  DIV & $s$, $i_{1}$, $i_{2}$ & $s$, $(i_{1}+i_{2})$ \\
  EQL & $s$, $i_{1}$, $i_{2}$ & $s$, $(i_{1}+i_{2})$ \\
  LTH & $s$, $i_{1}$, $i_{2}$ & $s$, $(i_{1}+i_{2})$ \\
  NAY & $s$, $i_{1}$, $i_{2}$ & $s$, $(i_{1}+i_{2})$ \\
  IF  & $s$, $b$, $e_{1}$, $e_{2}$ & $s$, (if $b$ $e_{1}$ else $e_{2}$) \\ \hline
\end{tabularx}
}
\caption{Instruction set}
\label{fig:inst_set}
\end{figure}

Looking at this, the problem of stack underflow becomes very apparent. Each instruction seems to require that very specific arguments are on the stack and since this is not encoded in the instruction, these stack requirements are enforced at run time rather compile time. Using dependent types, this encoding can be achieved by representing instructions with a type indexed by its required stack. Just indexing instructions by their required stack, however, is not sufficient. When we later want to build programs from our instructions we wish to be able to chain instructions, creating a sequence for the stack machine to run. To be able to do this, an instruction must also be indexed by what it leaves on the stack. 

\begin{figure}
\centering{
\begin{tabularx}{\textwidth}{ | X X X | }
  \hline
  \textbf{Instruction} & \textbf{Required Stack} & \textbf{Stack Produced} \\ \hline
  PUSH $i$ & $0$ & $1$ \\ 
  ADD & $2$ & $1$ \\
  SUB & $2$ & $1$ \\
  MUL & $2$ & $1$ \\
  DIV & $2$ & $1$ \\
  EQL & $2$ & $1$ \\
  LTH & $2$ & $1$ \\
  NAY & $2$ & $1$ \\
  IF  & $3$ & $1$ \\ \hline

\end{tabularx}
}
\caption{Instruction set with stack effects}
\label{fig:inst_set_with_effect}
\end{figure}

\begin{figure}
\begin{alltt}
data Inst : Nat \(\rightarrow\) Nat \(\rightarrow\) Type where
  PUSH : Int \(\rightarrow\) Inst          s    (S s)
  ADD  :        Inst    (S (S s))  (S s)
  SUB  :        Inst    (S (S s))  (S s)
  MUL  :        Inst    (S (S s))  (S s)
  DIV  :        Inst    (S (S s))  (S s)
  EQL  :        Inst    (S (S s))  (S s)
  LTH  :        Inst    (S (S s))  (S s)
  NAY  :        Inst       (S s)   (S s)
  IF   :        Inst (S (S (S s))) (S s)
\end{alltt}
\caption{Idris implementation of instructions}
\label{fig:idris_inst_set}
\end{figure}

Since the stack effect of an instruction is a minimum effect, these indices must be universally quantified over natural numbers. This universal quantification is expressed in Figure~\ref{fig:idris_inst_set} by that i.e. the PUSH $i$ instruction is Inst s (S s) rather than Inst 0 1 as it seen in Figure~\ref{fig:inst_set_with_effect}. 

\subsubsection{Program Representation}
To avoid the underflow problem a program must also have its effect encoded. Similar to instruction this encoding can be done by indexing programs by its required stack. Since we also want to be able to chain programs, just like instructions, the resulting stack of running a program must also be encoded into the type. As such a program is indexed by its stack effect:

\begin{alltt}
data Prog : Nat \(\rightarrow\)  Nat \(\rightarrow\)  Type where
\end{alltt}

Since a program is a list of instructions the value that programs are indexed by should be dictated by the instructions it contains. This means that when constructing programs they should take base in the instructions that they contain. To do this we use a structure similar to a list, as seen in Figure~\ref{fig:idris_impl_of_prg}. An empty program ($Nil$) is a program that does not change the stack. Adding an instruction to an existing program ($::$) will construct a new program parametrized by the instructions stack consumption and the programs stack production. Since instruction indices are universally quantified any instruction will fit on any program, but the type of the program constructed will reflect the new stack effect. This means that it is possible to construct programs that will cause stack underflow, but this will be present in the type, hence the stack machine can require a supplied stack to be of a sufficient size to avoid this underflow. This will be further discussed in Section~\ref{sec:running_a_program}.

\begin{figure}
\begin{alltt}
data Prog : Nat \(\rightarrow\)  Nat \(\rightarrow\)  Type where
  Nil  : Prog s s
  (::) : Inst s s\('\) \(\rightarrow\)  Prog s\('\) s\('\)\('\) \(\rightarrow\)  Prog s s\('\)\('\)
\end{alltt}
\caption{Idris representation of well-stacked programs}
\label{fig:idris_impl_of_prg}
\end{figure}

In addition to constructing programs from scratch, it can also be useful to concatenate them. Again we draw on our programs similarities with lists and use a similar name and notation to the $append$ (here \texttt{+++}) operation seen in many functional programming languages.

\begin{alltt}
(+++) : Prog s s\('\) -> Prog s\('\) s\('\)\('\) -> Prog s s\('\)\('\)
(+++) Nil p2       = p2
(+++) (i :: p1) p2 = i :: (p1 +++ p2)
\end{alltt}

The type of $append$ dictates that given two programs with stack effects $(s, s')$ and $(s', s'')$ we can construct a program with stack effect $(s, s'')$. As they are also implicit arguments to the $append$ function, any two programs can be chained together, if the Idris type checker can infer the arguments.
%!TEX root = ../main.tex
\subsection{Compile}
Now that we have a representation for stack machine programs that tracks stack effects, we can translate our well-typed expression language to well-stacked programs. The compiler should translate an \texttt{Expr} to a program that transforms a zero-element stack and into to an $n$-element stack, with the remaining stack being the result of evaluation. Since programs are indexed by their stack effect there should be a correspondence between the stack effect of the returned program and the type of the input expression. This should be expressed by the type of our \texttt{compile} function, for which we need to be able to construct program types with a stack effect based on arbitrary conditions, such as the type of the input expression. For this we use the follow representation of stack effects:

\begin{alltt}
data Eff : Type where
	Inc  : Nat \(\rightarrow\)  Eff
	Dec  : Nat \(\rightarrow\)  Eff
	Flat :        Eff

getProg : Nat \(\rightarrow\)  Eff \(\rightarrow\)  Type
getProg n (Inc m) = Prog n (S (m + n))
getProg n (Dec m) = Prog (S (m + n)) n
getProg n Flat    = Prog n n
\end{alltt}

The data type \texttt{Eff} represents a change in a value. It can either be increased (\texttt{Inc}), decreased (\texttt{Dec}), or remain unaltered (\texttt{Flat}). This is used to construct a program type with \texttt{getProg}, which based on a \texttt{Nat n} and an \texttt{Eff e} results in a program type requiring n elements, and leaving an amount based on the \texttt{Eff} on the stack. It is worth noting that \texttt{getProg} treats the \texttt{Inc} and \texttt{Dec} constructors as being zero indexed. This means that \texttt{getProg 0 (Inc 0)} will result in \texttt{Prog 0 1}. With this we can construct program types that are based on a starting state and an effect.

In order to use this for constructing programs based on expression types, all we have to do is write a function translating a \texttt{Tip} to an \texttt{Eff}:

\begin{alltt}
partial
getEff : Tip \(\rightarrow\) Eff
getEff TipUnit = Flat
getEff TipInt  = Inc 0
getEff TipBool = Inc 0
\end{alltt}

Note that \texttt{getEff} is partial. This is because our stack machine does not cover our entire expression language. This will be further discussion in Section~\ref{sec:omissions}.

With this we can create a type of program based on a \texttt{Tip t} requiring $s$ elements with \texttt{getProg s (getEff t)}, which allows us to compute the stack effect of a program from the type of its source expression.

Since our expression language contains lambda functions and applications our \texttt{compile} must be able to remember the applied expression and then substitute it in when appropriate. This means that we need some sort of stack frame to store these expressions. We know that there is a correspondence between the size of the stack frame and the type environment of the expression, so this should be enforced. At first glance a vector of expressions might seem like an obvious solution. However, this does not work, because each element in the vector could have a different type. This could be solved by using a \emph{sum type} (or \emph{dependent pair})\,\cite[p. 14]{Brady:IdrisTutorial}, by essentially having the type of each expression being existentially quantified. This, nevertheless, does not seem like an ideal solution. Representing the stack frame as a vector of programs yields the same issue.

To solve this we must take advantage of another correspondence between the stack frame and the environment. Not only do the two have the same size, \todo{Unclear?}but since our expression language only works on closed terms we know that there is a one to one correspondence between the type in the environment and the type of the expression in the stack frame. We can enforce this statically with the following data type:

\begin{alltt}
data StackFrame : Vect n Tip -> Nat -> Type where
	Nill : StackFrame [] Z
	(:::) : Expr G t -> StackFrame G n -> StackFrame (t :: G) (S n)
\end{alltt}

With this type a stack frame, the relationship between the expression environment and the type of expressions in the stack frame is enforced. As such, compiling a program means given an expression of type $t$, a stack frame matching the expressions environment, will result in a program matching the stack effect of the type, giving us the signature seen in Figure~\ref{fig:compile_function}. It important to note that not all parts of the expression language is covered by our compile function. This is as previously mentioned because our stack machine does not cover all parts of the expression language. Similarly to constructing and appending programs, the $s$ is an implicit argument, allowing the type checker to find a value for $s$ fitting the effect of the type.

\begin{figure}
\begin{alltt}
partial
compile : Expr G t -> StackFrame G n -> getProg s (getEff t)
compile U                 sf    = StackMachine4.Nil
compile (Val i)           sf    = [PUSH i]
compile (Boo True)        sf    = [PUSH 1]
compile (Boo False)       sf    = [PUSH 0]
compile (OpU Nay v)       sf    = compile v sf +++ [NAY]
compile (OpB o v1 v2) sf = compileOp o v1 v2 sf where
  partial 
  compileOp : BinOp a b c -> Expr G a -> Expr G b -> 
                StackFrame G n -> getProg s (getEff c)
  compileOp Add e1 e2 sf = compile e1 sf +++ compile e2 sf +++ [ADD]
  compileOp Sub e1 e2 sf = compile e1 sf +++ compile e2 sf +++ [SUB]
  compileOp Mul e1 e2 sf = compile e1 sf +++ compile e2 sf +++ [MUL]
  compileOp Div e1 e2 sf = compile e1 sf +++ compile e2 sf +++ [DIV]
  compileOp Eql e1 e2 sf = compile e1 sf +++ compile e2 sf +++ [EQL]
  compileOp Lt  e1 e2 sf = compile e1 sf +++ compile e2 sf +++ [LTH]
compile (If b tb fb) {t} sf with (t)
  | TipUnit = []
  | TipBool = compile tb sf +++ compile fb sf +++ compile b  sf +++ [IF]
  | TipInt  = compile tb sf +++ compile fb sf +++ compile b  sf +++ [IF]
compile (App (Lam b) e) sf = compile b (e ::: sf)
compile (Var stop) {G = x :: xs} (e ::: sf) = compile e sf
compile (Var (pop k)) (e ::: sf) = compile (Var k) sf
\end{alltt}
\label{fig:compile_function}
\caption{Our compile function. It is partial as it does not cover all constructors of \texttt{Expr}. \texttt{compileOp} is partial since it calls \texttt{compile}.}
\end{figure}
%!TEX root = ../main.tex
\subsection{Running a program}
\label{sec:running_a_program}
% Running a program
Now that we have a way of representing programs, and a function for converting our expression language to this representation ($compile$), we need a way of executing a program. A function running a program must use the information encoded into the indices of the type of the program to ensure no stack underflow. This means that no program should be runnable without a supplied stack satisfying the programs requirements. As such we can define an invariant dictating that a function $run$ taking a program $Prog\;s\;s'$, and a stack of size $s$ will result in a stack of size $s'$. This invariant can be seen in Figure~\ref{fig:run_function}. We use the build-in vector type from Idris to represent the stack as this allows us to specify the size of the stack. The items on the stack are, as previously mentioned all $integers$\todo{Actually mention this, and specify what kind of integer}. 

\begin{figure}
\begin{alltt}
run : Prog s s' -> Vect s Int -> Vect s' Int
run (PUSH v :: is) vs               = run is (v :: vs)
run (ADD    :: is) (v1 :: v2 :: vs) = run is ((v1 + v2) :: vs)
run (SUB    :: is) (v1 :: v2 :: vs) = run is ((v2 - v1) :: vs)
run (MUL    :: is) (v1 :: v2 :: vs) = run is ((v1 * v2) :: vs)
run (DIV    :: is) (v1 :: v2 :: vs) = run is ((cast ((cast v2) / (cast v1))) :: vs)
run (EQL    :: is) (v1 :: v2 :: vs) = let b = case (v1 == v2) of
                                                   True  => 1
                                                   False => 0
                                              in run is (b :: vs)
run (LTH    :: is) (v1 :: v2 :: vs) = let b = case (v1 < v2) of
                                                   True  => 1
                                                   False => 0
                                              in run is (b :: vs)
run (NAY    :: is)        (v :: vs) = let b = case v of
                                                   0 => 1
                                                   _ => 0
                                              in run is (b :: vs)
run (IF     :: is)        (b :: e1 :: e2 :: vs) = let v = case b of
                                                   0 => e1
                                                   _ => e2
                                              in run is (v :: vs)
run []             vs                 = vs
\end{alltt}
\caption{Our run function. The first argument is the function to be run, and the second argument is the stack for the program to be run on. The result is the stack after the program has been run.}
\label{fig:run_function}
\end{figure}

Since the first instruction in the program defines the programs stack requirement (by the definition of programs) and $run$ looks at one instruction at a time, the type of $run$, saying that the stack is a vector with a size that satisfies the programs requirement, ensures that it will never cause a stack underflow.
%!TEX root = ../main.tex
\subsection{Omissions}
\label{sec:omissions}
As previously mentioned, our \texttt{compile} function, and thereby our stack machine, does not cover our entire programming language. Specifically sum types, product types, and fixed points are not handled. 

\paragraph{Sum and Product Types}
When defining the instructions corresponding to sums and products, we run into a problem. This is because the stack effect of these terms depend on their contents. All other instructions always have the same effect. We need some way to represent instructions with variable stack effects.

\paragraph{Fixed points}
Compiling a fixed point for a stack machine that is not well-stacked is fairly straightforward. If treated like a function call, we could simply have a \texttt{CALL} instruction pointing to the start of the function with a conditional. With our well-stacked machine this is not as simple, since a \texttt{CALL} instruction in our representation would always have the same stack effect. Once again, we are in need of some way of representing instructions with varying stack effects.

\subsubsection{Possible Solution}
Both of the above issues could be solved by having an instruction where the effect is based on some parameter. We can use our previously defined \texttt{Eff} type for this. Consider the two functions, \texttt{stackReq} and \texttt{stackProduce}, in Figure~\ref{fig:stack_effect_functions}. These functions produce a natural number based on an \texttt{Eff} and a natural number. These can be used to construct a variable instruction based on an \texttt{Eff} also seen in Figure~\ref{fig:stack_effect_functions}. 

\begin{figure}
\begin{alltt}
stackProduce : Eff \(\to\) Nat \(\to\) Nat
stackProduce Flat    n = n
stackProduce (Inc Z) n = S n
stackProduce (Inc m) n = S (m+n)
stackProduce (Dec _) n = n

stackReq : Eff \(\to\) Nat \(\to\) Nat
stackReq Flat    n = n
stackReq (Inc x) n = n
stackReq (Dec Z) n = S n
stackReq (Dec m) n = S (m+n)
\end{alltt}
\caption{Functions creating natural numbers from \texttt{Eff}s used for indexing instructions.}
\label{fig:stack_effect_functions}
\end{figure}

We could use these functions to create a \texttt{CALL} instruction with a stack effect based on an \texttt{Eff}. This \texttt{CALL} instruction could be used to implement function calls, and thereby fixed points from our well-typed representation. To do this we would need an additional layer of abstraction to our current program representation. This could be a vector containing multiple instances of our current program representation, where each element in the vector can be thought of as a function. For this to work we need to encode more information into the \texttt{CALL} instruction. Firstly, we need a program environment which contains the programs this instruction can \emph{call}. Secondly, we need an index into this environment to specify what function to call.

In Section~\ref{sec:a-well-typed-expression-language} we describe how variables are represented in our well-typed representation by using a predicate. We can use the same approach for function calls, by having our \texttt{CALL} instruction parameterized by a membership predicate on its enclosing environment. This can be used as a intrinsic proof that a program at a given position in the environment has a specific stack effect. 

\begin{figure}
\begin{alltt}
data HasEffect : Vect n Eff \(\to\) Eff \(\to\) Type where
    stop : HasEffect (e :: E) e
    pop  : HasEffect E e \(\to\) HasEffect (f :: E) e

data Inst : Nat \(\to\) Nat \(\to\) Type where
    \vdots    
    CALL : (e : Eff) \(\to\) HasEffect E e \(\to\) 
           Inst (stackReq e s) (stackProduce e s)
\end{alltt}
\caption{\texttt{HasEffect} membership predicate and \texttt{CALL} instruction with variable indices based on an \texttt{Eff}.}
\label{fig:call}
\end{figure}

Figure~\ref{fig:call} shows this membership predicate (\texttt{HasEffect}) and an instruction used for function calls. For constructing a \texttt{CALL} instruction we need an \texttt{Eff} and our membership predicate. For the sake of simplicity we will henceforth use a numerical representation for the index (i.e. \texttt{stop} is \texttt{0}, \texttt{pop stop} is \texttt{1} etc.). The intuition behind \texttt{HasEffect} is analogous to the one behind \texttt{HasType} in the well-typed representation.

\begin{figure}
\begin{alltt}
p 0: [PUSH 2, CALL (Inc 0) 1, ADD]
     ::
p 1: [PUSH 1, CALL (Inc 0) 2, ADD]
     ::
p 2: Nil
     ::
p 3: [PUSH 4]
     ::
p 4: Nil
\end{alltt}
\caption{Possible program structure with functions. p 0--4 are the programs in the program vector.}
\label{fig:program_structure}
\end{figure}

In Figure~\ref{fig:program_structure}, \texttt{p 0} is the main program. \texttt{CALL (Inc 0) 1} indicates a call to the next program in the program vector with stack effect \texttt{(s,~S~s)}, \texttt{p 1}. \texttt{p 1} contains the instruction \texttt{CALL (Inc 0) 2}, which indicates another call, this time to the program two elements further in the program vector. Note that this is a call to \texttt{p 3} as the parameter of \texttt{CALL} is relative to the current program, and not a global index in the program vector. 

Compiling a fixed point term from Figure~\ref{fig:idris-def-expr-lang} (\texttt{Fix e}) would mean constructing a \texttt{CALL} instruction based on the type of the term, after which the term \texttt{e} is compiled and added to both the program vector and the environment. Since the type of the term being compiled determines the stack effect of the resulting program, the effect of the \texttt{CALL} instruction and the compiled program will be identical. When running the program this correspondence between effects will allow us to run the function in place of the call, as their stack effects match.

Note that since we can only call programs further down the program vector, mutual recursion is not possible. Regular recursion is available through \texttt{CALL~e~0}.

Interestingly, the above way of handling function calls also be used for product and sum types. For example, treating each part of a product type as a function, and \texttt{Fst} and \texttt{Snd} as function calls would be one way to implement this.

It is worth noting that this has not been fully implemented due to time constraints. The shown implementations of \texttt{stackReq}, \texttt{stackProduce}, \texttt{HasEffect}, and \texttt{CALL} are all accepted by the Idris type checker, but there might be complications when completing the implementation requiring alterations to these definitions. Furthermore, fully implementing these changes would involve revamping many parts of our stack machine implementation to handle a vector of programs rather than just a program.

