%!TEX root = ../main.tex
\section{Stack Machine}
\label{sec:stack-machine}

% - Hvad er en stack machine?
% - Udfordringer
% - Løsninger
% - Program representation
% - Compilation
% - Execution

Now that we can produce terms in our well-typed representation, we would like to go a step further. This means compiling these terms to some portable representation which can then be executed, while using dependent types to retain as much safety as possible. To accomplish this, we need a compiler to translate terms in the well-typed representation to instructions, and a machine to run these instructions. Due to its simplicity, we have chosen to implement a stack machine.

A minimal stack machine\,\cite[pp. 157--161]{Sestoft:PLC} needs only a few components: a list of instructions, a program counter (PC), a stack, and a stack pointer (SP). The program counter records how far into the list of instructions the program is, and the stack pointer points to the topmost element of the stack. As each instruction is executed, the stack is manipulated and the program counter is increased. Because each instruction operates on the top of the stack, programs are written in a postfix style. For example, to add the constants 3 and 4 together, the instructions required would be: \texttt{[PUSH 3, PUSH 4, ADD]}. The first two instructions push the constants onto the stack, and the third instruction adds them and leaves the result on the stack. Thus each \texttt{PUSH} instruction has the stack effect of adding one value to the stack, and the effect of the \texttt{ADD} instruction is to consume two values from the stack, and leave one value. The stack effect of the two instructions can be written (0, 1) and (2, 1), respectively.

\subsubsection{Problems}
In most stack machine implementations, many errors can occur at run time. For example, running the program \texttt{[PUSH 3, ADD]} with an empty stack will result in a stack underflow error, as the \texttt{ADD} instruction is expecting 2 values on the stack. It is up to the compiler to make sure all the instructions are correct. Unfortunately, most compilers are not accompanied by a proof of correctness. Instead they are thoroughly tested. As a consequence, even if you have proved the correctness of your program, as soon as you compile it with an unverified compiler, you lose all assurances, as Xavier Leroy points out in an article on the verified C compiler CompCert\,\cite{Leroy_formalverification}.

\subsection{Desired Qualities}
While it is beyond the scope of this project to produce a fully verified compiler and stack machine, we can use dependent types to make certain guarantees about the compiler, the programs it produces, and the stack machine that runs the programs. Specifically, we would like a program representation that encodes the stack effect of the instructions that comprise it, and a stack machine that only accepts \textit{well-stacked} programs that will run with an empty stack, without any stack underflow errors. The stack machine will be very simple. Instead of using a program counter and a stack pointer, the program and the stack themselves will be passed as parameters to the stack machine. As the stack machine is only used to run programs compiled from the well-typed representation, which has no terms requiring a heap, all values will be placed on the stack. This means a heap is not needed, and was not implemented.

%!TEX root = ../main.tex
\subsection{Programs}
\label{sec:program}
As previously mentioned a program run by a stack machine is a list of instructions. Our machine's instruction set can be seen in Figure~\ref{fig:inst_set}. Most instructions should be self-explanatory, but for further explanation our stack machine is inspired by Peter Sestofts abstract machine for micro-C\,\cite[pp. 157--161]{Sestoft:PLC}. When executed, an instructions has an effect on the stack specific to each instruction.

\begin{figure}
\centering{
\begin{tabularx}{\textwidth}{ | X X X | }
  \hline
  \textbf{Instruction} & \textbf{Stack Before} & \textbf{Stack After} \\ \hline
  PUSH $i$ & $s$ & $s$, $i$ \\ 
  ADD & $s$, $i_{1}$, $i_{2}$ & $s$, $(i_{1}+i_{2})$ \\
  SUB & $s$, $i_{1}$, $i_{2}$ & $s$, $(i_{1}+i_{2})$ \\
  MUL & $s$, $i_{1}$, $i_{2}$ & $s$, $(i_{1}+i_{2})$ \\
  DIV & $s$, $i_{1}$, $i_{2}$ & $s$, $(i_{1}+i_{2})$ \\
  EQL & $s$, $i_{1}$, $i_{2}$ & $s$, $(i_{1}+i_{2})$ \\
  LTH & $s$, $i_{1}$, $i_{2}$ & $s$, $(i_{1}+i_{2})$ \\
  NAY & $s$, $i_{1}$, $i_{2}$ & $s$, $(i_{1}+i_{2})$ \\
  IF  & $s$, $b$, $e_{1}$, $e_{2}$ & $s$, (if $b$ $e_{1}$ else $e_{2}$) \\ \hline
\end{tabularx}
}
\caption{Instruction set}
\label{fig:inst_set}
\end{figure}

Looking at this, the problem of stack underflow becomes very apparent. Each instruction seems to require that very specific arguments are on the stack and since this is not encoded in the instruction, these stack requirements are enforced at run time rather compile time. Using dependent types, this encoding can be achieved by representing instructions with a type indexed by its required stack. Just indexing instructions by their required stack, however, is not sufficient. When we later want to build programs from our instructions we wish to be able to chain instructions, creating a sequence for the stack machine to run. To be able to do this, an instruction must also be indexed by what it leaves on the stack. 

\begin{figure}
\centering{
\begin{tabularx}{\textwidth}{ | X X X | }
  \hline
  \textbf{Instruction} & \textbf{Required Stack} & \textbf{Stack Produced} \\ \hline
  PUSH $i$ & $0$ & $1$ \\ 
  ADD & $2$ & $1$ \\
  SUB & $2$ & $1$ \\
  MUL & $2$ & $1$ \\
  DIV & $2$ & $1$ \\
  EQL & $2$ & $1$ \\
  LTH & $2$ & $1$ \\
  NAY & $2$ & $1$ \\
  IF  & $3$ & $1$ \\ \hline

\end{tabularx}
}
\caption{Instruction set with stack effects}
\label{fig:inst_set_with_effect}
\end{figure}

\begin{figure}
\begin{alltt}
data Inst : Nat \(\rightarrow\) Nat \(\rightarrow\) Type where
  PUSH : Int \(\rightarrow\) Inst          s    (S s)
  ADD  :        Inst    (S (S s))  (S s)
  SUB  :        Inst    (S (S s))  (S s)
  MUL  :        Inst    (S (S s))  (S s)
  DIV  :        Inst    (S (S s))  (S s)
  EQL  :        Inst    (S (S s))  (S s)
  LTH  :        Inst    (S (S s))  (S s)
  NAY  :        Inst       (S s)   (S s)
  IF   :        Inst (S (S (S s))) (S s)
\end{alltt}
\caption{Idris implementation of instructions}
\label{fig:idris_inst_set}
\end{figure}

Since the stack effect of an instruction is a minimum effect, these indices must be universally quantified over natural numbers. This universal quantification is expressed in Figure~\ref{fig:idris_inst_set} by that i.e. the PUSH $i$ instruction is Inst s (S s) rather than Inst 0 1 as it seen in Figure~\ref{fig:inst_set_with_effect}. 

\subsubsection{Program Representation}
To avoid the underflow problem a program must also have its effect encoded. Similar to instruction this encoding can be done by indexing programs by its required stack. Since we also want to be able to chain programs, just like instructions, the resulting stack of running a program must also be encoded into the type. As such a program is indexed by its stack effect:

\begin{alltt}
data Prog : Nat \(\rightarrow\)  Nat \(\rightarrow\)  Type where
\end{alltt}

Since a program is a list of instructions the value that programs are indexed by should be dictated by the instructions it contains. This means that when constructing programs they should take base in the instructions that they contain. To do this we use a structure similar to a list, as seen in Figure~\ref{fig:idris_impl_of_prg}. An empty program ($Nil$) is a program that does not change the stack. Adding an instruction to an existing program ($::$) will construct a new program parametrized by the instructions stack consumption and the programs stack production. Since instruction indices are universally quantified any instruction will fit on any program, but the type of the program constructed will reflect the new stack effect. This means that it is possible to construct programs that will cause stack underflow, but this will be present in the type, hence the stack machine can require a supplied stack to be of a sufficient size to avoid this underflow. This will be further discussed in Section~\ref{sec:running_a_program}.

\begin{figure}
\begin{alltt}
data Prog : Nat \(\rightarrow\)  Nat \(\rightarrow\)  Type where
  Nil  : Prog s s
  (::) : Inst s s\('\) \(\rightarrow\)  Prog s\('\) s\('\)\('\) \(\rightarrow\)  Prog s s\('\)\('\)
\end{alltt}
\caption{Idris representation of well-stacked programs}
\label{fig:idris_impl_of_prg}
\end{figure}

In addition to constructing programs from scratch, it can also be useful to concatenate them. Again we draw on our programs similarities with lists and use a similar name and notation to the $append$ (here \texttt{+++}) operation seen in many functional programming languages.

\begin{alltt}
(+++) : Prog s s\('\) -> Prog s\('\) s\('\)\('\) -> Prog s s\('\)\('\)
(+++) Nil p2       = p2
(+++) (i :: p1) p2 = i :: (p1 +++ p2)
\end{alltt}

The type of $append$ dictates that given two programs with stack effects $(s, s')$ and $(s', s'')$ we can construct a program with stack effect $(s, s'')$. As they are also implicit arguments to the $append$ function, any two programs can be chained together, if the Idris type checker can infer the arguments.
%!TEX root = ../main.tex
\subsection{Compile}

	First version: Expr G t -> Prog s (S s)
		Function Application
	Second version: Expr G t -> Vector n (Prog s (S s)) -> Prog s (S s)
		Programs not ending in +1
	Third version: Expr G t -> Vector n (Prog s (S s)) -> getProg s (getEff t)
		No Program Counter
	Fourth version: ??
%!TEX root = ../main.tex
\subsection{Running a program}
\label{sec:running_a_program}
% Running a program
Now that we have a way of representing programs, and a function for converting our expression language to this representation ($compile$), we need a way of executing a program. A function running a program must use the information encoded into the indices of the type of the program to ensure no stack underflow. This means that no program should be runnable without a supplied stack satisfying the programs requirements. As such we can define an invariant dictating that a function $run$ taking a program $Prog\;s\;s'$, and a stack of size $s$ will result in a stack of size $s'$. This invariant can be seen in Figure~\ref{fig:run_function}. We use the build-in vector type from Idris to represent the stack as this allows us to specify the size of the stack. The items on the stack are, as previously mentioned all $integers$\todo{Actually mention this}. 

\begin{figure}
\begin{alltt}
run : Prog s s' -> Vect s Int -> Vect s' Int
run (PUSH v :: is) vs               = run is (v :: vs)
run (ADD    :: is) (v1 :: v2 :: vs) = run is ((v1 + v2) :: vs)
run (SUB    :: is) (v1 :: v2 :: vs) = run is ((v2 - v1) :: vs)
run (MUL    :: is) (v1 :: v2 :: vs) = run is ((v1 * v2) :: vs)
run (DIV    :: is) (v1 :: v2 :: vs) = run is ((cast ((cast v2) / (cast v1))) :: vs)
run (EQL    :: is) (v1 :: v2 :: vs) = let b = case (v1 == v2) of
                                                   True  => 1
                                                   False => 0
                                              in run is (b :: vs)
run (LTH    :: is) (v1 :: v2 :: vs) = let b = case (v1 < v2) of
                                                   True  => 1
                                                   False => 0
                                              in run is (b :: vs)
run (NAY    :: is)        (v :: vs) = let b = case v of
                                                   0 => 1
                                                   _ => 0
                                              in run is (b :: vs)
run (IF     :: is)        (b :: e1 :: e2 :: vs) = let v = case b of
                                                   0 => e1
                                                   _ => e2
                                              in run is (v :: vs)
run []             vs                 = vs
\end{alltt}
\caption{Our run function. The first argument is the function to be run, and the second argument is the stack for the program to be run on. The result is the stack after the program has been run.}
\label{fig:run_function}
\end{figure}

Since the first instruction in the program defines the programs stack requirement (by the definition of programs) and $run$ looks at one instruction at a time, the type of $run$, saying that the stack is a vector with a size that satisfies the programs requirement, ensures that it will never cause a stack underflow.
%!TEX root = ../main.tex
\subsection{Omissions}
\label{sec:omissions}\todo{Update omissions to most recent work before hand-in}
As previously mentioned our \texttt{compile} function, and thereby our stack machine, does not cover our entire well-typed expression language. Specifically the sum type, product type, and fixed point representations are not handled. 

\paragraph{Sum and Product Types}
These could be handled in two different ways. Either run both branches, choose one, and throw the result of the other one away, or make the decision before running a branch. This is a matter of whether the operator that chooses a branch (i.e. \texttt{Fst} or \texttt{Snd}) is before or after the branches in the program. Both ways have complications when working with our well-stacked stack machine. 

The first will cause issues in terms of the stack effect of an instruction. Intuitively a simple pair of values \texttt{(a, b)} would have stack effect (0,2) and \texttt{Fst} and \texttt{Snd} would then have effect (2,1). This is inconsistent with our invariant when compiling which dictates that, compiling any expression will result in an instruction with an effect corresponding to the type of the compiled expression. Since i.e. \texttt{Fst} can have any type we cannot guarantee this.


Since the result of one branch will need to be thrown away, the instruction choosing a branch will have the effect of consuming one branch (the one thrown away). This is inconsistent with our invariant when compiling which dictates that, compiling any expression will result in an instruction with an effect corresponding to the type of the compiled expression. 

The former causes violations with the invariant when running, which states that running a program with stack effect $(s, s')$ on a stack of size $s$ must result in a stack of size $s'$. When throwing away the unwanted branch of the program, the resulting program might be inconsistent with the invariant since we have no way of knowing the stack effect of a program without a certain amount of instructions.

\paragraph{Fixed points}
Compiling a fixed point for a stack machine that is not well-stacked is fairly straightforward. If treated like a function call, simply have a \texttt{CALL} instruction pointing to the start of the function with a conditional. With our well-stacked machine this is not as simple. Since our instructions are indexed by their stack effect, what stack effect would such a \texttt{CALL} instruction have? 

\paragraph{Possible Solution}
Both of the above issues could be solved by having an instruction where the effect is based on some parameter. We can use our previously defined \texttt{Eff} type for this. Consider the two functions, \texttt{stackReq} and \texttt{stackProduce}, in Figure~\ref{fig:stack_effect_functions}. These functions produce a natural number based on an \texttt{Eff} and a natural number. These can be used to construct a variable instruction based on an \texttt{Eff} also seen in Figure~\ref{fig:stack_effect_functions}. 

\begin{figure}
\begin{alltt}
stackProduce : Eff -> Nat -> Nat
stackProduce Flat n = n
stackProduce (Inc Z) n = S n
stackProduce (Inc m) n = S (m+n)
stackProduce (Dec _) n = n

stackReq : Eff -> Nat -> Nat
stackReq Flat n = n
stackReq (Dec Z) n = S n
stackReq (Dec m) n = S (m+n)
stackReq (Inc x) n = n
\end{alltt}
\caption{Functions creating natural numbers from \texttt{Eff}s used for indexing instructions.}
\label{fig:stack_effect_functions}
\end{figure}

We can use these functions to create a \texttt{CALL} instruction with a stack effect based on an \texttt{Eff}. This \texttt{CALL} instruction can be used to implement function calls, and thereby fixed points from our expression language. To do this we need an additional layer of abstraction to our current program representation. We introduce a program vector which is a vector of \texttt{Prog s s'}s, where each element in the vector can be thought of as a function. For this to work we need to encode more information into the \texttt{CALL} instruction. Firstly we need a program environment which contains the programs this instruction can \emph{call}. Secondly we need an index into this environment to specify what function to call.

In Section~\ref{sec:a-well-typed-expression-language} we describe how variables are represented in our expression language by using a predicate, modeling De Bruijn indices. We can use the same approach for function calls as well, by having our \texttt{CALL} instruction parametrized by a membership predicate on its enclosing environment. This can be used as a intrinsic proof that a program at a given position in the environment has a given specific stack effect. 

\begin{figure}
\begin{alltt}
data HasEffect : Vect n Eff -> Eff -> Type where
    stop : HasEffect (e :: E) e
    pop  : HasEffect E e -> HasEffect (f :: E) e

CALL : (e : Eff) -> HasEffect E e -> Inst (stackReq e s) (stackProduce e s)
\end{alltt}
\caption{\texttt{HasEffect} membership predicate and \texttt{CALL} instruction with variable indices based on an \texttt{Eff}.}
\label{fig:call}
\end{figure}

Figure~\ref{fig:call} shows this membership predicate (\texttt{HasEffect}) and an instruction used for function calls. For constructing a \texttt{CALL} instruction we need an \texttt{Eff} and our membership predicate. For the sake of simplicity we will henceforth use a numerical representation for the index (i.e. \texttt{stop} is \texttt{0}, \texttt{pop stop} is \texttt{1} etc.).

\begin{figure}
\begin{alltt}
p 0: [PUSH 2, CALL (Inc 0) 1, ADD]
     ::
p 1: [PUSH 1, CALL (Inc 0) 2, ADD]
     ::
p 2: Nil
     ::
p 3: [PUSH 4]
     ::
p 4: Nil
\end{alltt}
\caption{Possible program structure with functions. p 0--5 are the programs in the program vector.}
\label{fig:program_structure}
\end{figure}

In the example Figure~\ref{fig:program_structure}, \texttt{p 0} is the main program. \texttt{CALL (Inc 0) 1} indicates a call to the next program in the program vector with stack effect \texttt{(s,~S~s)}, \texttt{p 1}. \texttt{p 1} contains the instruction \texttt{CALL (Inc 0) 2}, which indicates another call this time to the program two further down the program vector. Note that this is a call to \texttt{p 3} as \texttt{CALL}s parameter is relative to the current program, and not a global index in the program vector. 

Compiling a fixed point expression from Figure~\ref{fig:idris-def-expr-lang} (\texttt{Fix e}) would therefore mean constructing a \texttt{CALL} instruction based on type of the expression, after which the expression \texttt{e} is compiled and added to both the program vector, and environment. Since the type of the expression when compiled gives the resulting program its stack effect, the effect of the \texttt{CALL} instruction and the compiled program will be identical. When running the program this correspondence between effects will allow us to run the function in the place of a call, as their stack effects match.

To ensure that \texttt{CALL} will not refer to a program not existing in the program instructions must also be indexed by the context they exist in. This way it can be enforced that the \texttt{Nat} in the \texttt{CALL} constructor will not be higher than the size of the program vector, possibly using a finite set instead of a \texttt{Nat}.

Note that since we can only call programs further down the program vector, mutual recursion is not possible. Regular recursion is available through \texttt{CALL~e~0}.

Interestingly enough the above way of handling function calls also works for product and sum types. For example, treating each part of a product types as a function, and then \texttt{Fst} and \texttt{Snd} as function calls would be a way to implement this.

It is worth noting that this has not been fully implemented due to time constraints. The shown implementations of \texttt{stackReq}, \texttt{stackProduce}, \texttt{HasEffect}, and \texttt{CALL} are all accepted by the Idris type checker, but there might be complications when implementing the rest of this requiring alterations to these. Furthermore, fully implementing this would involve revamping many parts of our stack machine implementation to handle a program vector rather than just a program.
