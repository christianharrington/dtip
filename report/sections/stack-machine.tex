%!TEX root = ../main.tex
\section{Stack Machine}
\label{sec:stack-machine}
Now that we have a well-typed expression language, we would like to be able to compile and execute it. This means we need a compiler to translate the expression language to instructions, and a stack machine to run these instructions.

\subsection{Standard Stack Machine}
A minimal stack machine needs only a few components: a list of instructions, a program counter (PC), a stack, and a stack pointer (SP). The program counter records how far into the array of instructions we are, and the stack pointer points to the topmost element of the stack. As each instruction is evaluated, the stack is manipulated and the program counter is increased. The instructions are usually encoded using postfix notation. For example, if you want to add the constants 3 and 4 together, the instructions required would be: [PUSH 3, PUSH 4, ADD]. The first two instructions push the constants onto the stack, and the third instruction adds them together, and leaves the result on the stack.

\subsubsection{Problems}
In most implementations, the stack machine is not in it self very safe. It is up to the compiler to make sure all the instructions are correct. But there is nothing in the program representation or the stack machine stopping the compiler from producing impossible programs. For example, the program [PUSH 3, ADD] will result in a stack underflow error, as the ADD instruction is expecting 2 values on the stack.

\subsection{Desired Qualities}
Since our expression language is safe, we would like our compiled programs to retain as much of that safety as possible. In this case, we will ensure the compiler can never produce a program with stack underflow errors.

% What would we like in our stack machine. Our language is safe -> our bytecode should be safe. We can do this with DT
% Christian
%!TEX root = ../main.tex
\subsection{Programs}
\label{sec:program}
As previously mentioned a program run by a stack machine is a list of instructions. Our machine's instruction set can be seen in Figure~\ref{fig:inst_set}. Most instructions should be self-explanatory, but for further explanation our stack machine is inspired by Peter Sestofts abstract machine for micro-C\,\cite[pp. 157-161]{Sestoft:PLC}. When executed, an instructions has an effect on the stack specific to each instruction.

\begin{figure}
\centering{
\begin{tabularx}{\textwidth}{ | X X X | }
  \hline
  \textbf{Instruction} & \textbf{Stack Before} & \textbf{Stack After} \\ \hline
  PUSH $i$ & $s$ & $s$, $i$ \\ 
  ADD & $s$, $i_{1}$, $i_{2}$ & $s$, $(i_{1}+i_{2})$ \\
  SUB & $s$, $i_{1}$, $i_{2}$ & $s$, $(i_{1}+i_{2})$ \\
  MUL & $s$, $i_{1}$, $i_{2}$ & $s$, $(i_{1}+i_{2})$ \\
  DIV & $s$, $i_{1}$, $i_{2}$ & $s$, $(i_{1}+i_{2})$ \\
  EQL & $s$, $i_{1}$, $i_{2}$ & $s$, $(i_{1}+i_{2})$ \\
  LTH & $s$, $i_{1}$, $i_{2}$ & $s$, $(i_{1}+i_{2})$ \\
  NAY & $s$, $i_{1}$, $i_{2}$ & $s$, $(i_{1}+i_{2})$ \\
  IF  & $s$, $b$, $e_{1}$, $e_{2}$ & $s$, (if $b$ $e_{1}$ else $e_{2}$) \\ \hline
\end{tabularx}
}
\caption{Instruction set}
\label{fig:inst_set}
\end{figure}

Looking at this, the problem of stack underflow becomes very apparent. Each instruction seems to require that very specific arguments are on the stack and since this is not encoded in the instruction, these stack requirements are enforced at run time rather compile time. Using dependent types, this encoding can be achieved by representing instructions with a type indexed by its required stack. Just indexing instructions by their required stack, however, is not sufficient. When we later want to build programs from our instructions we wish to be able to chain instructions, creating a sequence for the stack machine to run. To be able to do this, an instruction must also be indexed by what it leaves on the stack. 

\begin{figure}
\centering{
\begin{tabularx}{\textwidth}{ | X X X | }
  \hline
  \textbf{Instruction} & \textbf{Required Stack} & \textbf{Stack Produced} \\ \hline
  PUSH $i$ & $0$ & $1$ \\ 
  ADD & $2$ & $1$ \\
  SUB & $2$ & $1$ \\
  MUL & $2$ & $1$ \\
  DIV & $2$ & $1$ \\
  EQL & $2$ & $1$ \\
  LTH & $2$ & $1$ \\
  NAY & $2$ & $1$ \\
  IF  & $3$ & $1$ \\ \hline

\end{tabularx}
}
\caption{Instruction set with stack effects}
\label{fig:inst_set_with_effect}
\end{figure}

\begin{figure}
\begin{alltt}
data Inst : Nat \(\rightarrow\) Nat \(\rightarrow\) Type where
  PUSH : Int \(\rightarrow\) Inst          s    (S s)
  ADD  :        Inst    (S (S s))  (S s)
  SUB  :        Inst    (S (S s))  (S s)
  MUL  :        Inst    (S (S s))  (S s)
  DIV  :        Inst    (S (S s))  (S s)
  EQL  :        Inst    (S (S s))  (S s)
  LTH  :        Inst    (S (S s))  (S s)
  NAY  :        Inst       (S s)   (S s)
  IF   :        Inst (S (S (S s))) (S s)
\end{alltt}
\caption{Idris implementation of instructions}
\label{fig:idris_inst_set}
\end{figure}

Since the stack effect of an instruction is a minimum effect, these indices must be universally quantified over natural numbers. This universal quantification is expressed in Figure~\ref{fig:idris_inst_set} by that i.e. the PUSH $i$ instruction is Inst s (S s) rather than Inst 0 1 as it seen in Figure~\ref{fig:inst_set_with_effect}. 

\subsubsection{Program Representation}
To avoid the underflow problem a program must also have its effect encoded. Similar to instruction this encoding can be done by indexing programs by its required stack. Since we also want to be able to chain programs, just like instructions, the resulting stack of running a program must also be encoded into the type. As such a program is indexed by its stack effect:

\begin{alltt}
data Prog : Nat \(\rightarrow\)  Nat \(\rightarrow\)  Type where
\end{alltt}

Since a program is a list of instructions the value that programs are indexed by should be dictated by the instructions it contains. This means that when constructing programs they should take base in the instructions that they contain. To do this we use a structure similar to a list, as seen in Figure~\ref{fig:idris_impl_of_prg}. An empty program ($Nil$) is a program that does not change the stack. Adding an instruction to an existing program ($::$) will construct a new program parametrized by the instructions stack consumption and the programs stack production. Since instruction indices are universally quantified any instruction will fit on any program, but the type of the program constructed will reflect the new stack effect. This means that it is possible to construct programs that will cause stack underflow, but this will be present in the type, hence the stack machine can require a supplied stack to be of a sufficient size to avoid this underflow. This will be further discussed in Section~\ref{sec:running_a_program}.

\begin{figure}
\begin{alltt}
data Prog : Nat \(\rightarrow\)  Nat \(\rightarrow\)  Type where
  Nil  : Prog s s
  (::) : Inst s s\('\) \(\rightarrow\)  Prog s\('\) s\('\)\('\) \(\rightarrow\)  Prog s s\('\)\('\)
\end{alltt}
\caption{Idris representation of well-stacked programs}
\label{fig:idris_impl_of_prg}
\end{figure}

In addition to constructing programs from scratch, it can also be useful to concatenate them. Again we draw on our programs similarities with lists and use a similar name and notation to the $append$ (here \texttt{+++}) operation seen in many functional programming languages.

\begin{alltt}
(+++) : Prog s s\('\) -> Prog s\('\) s\('\)\('\) -> Prog s s\('\)\('\)
(+++) Nil p2       = p2
(+++) (i :: p1) p2 = i :: (p1 +++ p2)
\end{alltt}

The type of $append$ dictates that given two programs with stack effects $(s, s')$ and $(s', s'')$ we can construct a program with stack effect $(s, s'')$. As they are also implicit arguments to the $append$ function, any two programs can be chained together, if the Idris type checker can infer the arguments.
%!TEX root = ../main.tex
\subsection{Compile}
Now that we have a representation for stack machine programs that tracks stack effects, we can translate our well-typed expression language to well-stacked programs. The compiler should translate an \texttt{Expr} to a program that transforms a zero-element stack and into to an $n$-element stack, with the remaining stack being the result of evaluation. Since programs are indexed by their stack effect there should be a correspondence between the stack effect of the returned program and the type of the input expression. This should be expressed by the type of our \texttt{compile} function, for which we need to be able to construct program types with a stack effect based on arbitrary conditions, such as the type of the input expression. For this we use the follow representation of stack effects:

\begin{alltt}
data Eff : Type where
	Inc  : Nat \(\rightarrow\)  Eff
	Dec  : Nat \(\rightarrow\)  Eff
	Flat :        Eff

getProg : Nat \(\rightarrow\)  Eff \(\rightarrow\)  Type
getProg n (Inc m) = Prog n (S (m + n))
getProg n (Dec m) = Prog (S (m + n)) n
getProg n Flat    = Prog n n
\end{alltt}

The data type \texttt{Eff} represents a change in a value. It can either be increased (\texttt{Inc}), decreased (\texttt{Dec}), or remain unaltered (\texttt{Flat}). This is used to construct a program type with \texttt{getProg}, which based on a \texttt{Nat n} and an \texttt{Eff e} results in a program type requiring n elements, and leaving an amount based on the \texttt{Eff} on the stack. It is worth noting that \texttt{getProg} treats the \texttt{Inc} and \texttt{Dec} constructors as being zero indexed. This means that \texttt{getProg 0 (Inc 0)} will result in \texttt{Prog 0 1}. With this we can construct program types that are based on a starting state and an effect.

In order to use this for constructing programs based on expression types, all we have to do is write a function translating a \texttt{Tip} to an \texttt{Eff}:

\begin{alltt}
partial
getEff : Tip \(\rightarrow\) Eff
getEff TipUnit = Flat
getEff TipInt  = Inc 0
getEff TipBool = Inc 0
\end{alltt}

Note that \texttt{getEff} is partial. This is because our stack machine does not cover our entire expression language. This will be further discussion in Section~\ref{sec:omissions}.

With this we can create a type of program based on a \texttt{Tip t} requiring $s$ elements with \texttt{getProg s (getEff t)}, which allows us to compute the stack effect of a program from the type of its source expression.

Since our expression language contains lambda functions and applications our \texttt{compile} must be able to remember the applied expression and then substitute it in when appropriate. This means that we need some sort of stack frame to store these expressions. We know that there is a correspondence between the size of the stack frame and the type environment of the expression, so this should be enforced. At first glance a vector of expressions might seem like an obvious solution. However, this does not work, because each element in the vector could have a different type. This could be solved by using a \emph{sum type} (or \emph{dependent pair})\,\cite[p. 14]{Brady:IdrisTutorial}, by essentially having the type of each expression being existentially quantified. This, nevertheless, does not seem like an ideal solution. Representing the stack frame as a vector of programs yields the same issue.

To solve this we must take advantage of another correspondence between the stack frame and the environment. Not only do the two have the same size, \todo{Unclear?}but since our expression language only works on closed terms we know that there is a one to one correspondence between the type in the environment and the type of the expression in the stack frame. We can enforce this statically with the following data type:

\begin{alltt}
data StackFrame : Vect n Tip -> Nat -> Type where
	Nill : StackFrame [] Z
	(:::) : Expr G t -> StackFrame G n -> StackFrame (t :: G) (S n)
\end{alltt}

With this type a stack frame, the relationship between the expression environment and the type of expressions in the stack frame is enforced. As such, compiling a program means given an expression of type $t$, a stack frame matching the expressions environment, will result in a program matching the stack effect of the type, giving us the signature seen in Figure~\ref{fig:compile_function}. It important to note that not all parts of the expression language is covered by our compile function. This is as previously mentioned because our stack machine does not cover all parts of the expression language. Similarly to constructing and appending programs, the $s$ is an implicit argument, allowing the type checker to find a value for $s$ fitting the effect of the type.

\begin{figure}
\begin{alltt}
partial
compile : Expr G t -> StackFrame G n -> getProg s (getEff t)
compile U                 sf    = StackMachine4.Nil
compile (Val i)           sf    = [PUSH i]
compile (Boo True)        sf    = [PUSH 1]
compile (Boo False)       sf    = [PUSH 0]
compile (OpU Nay v)       sf    = compile v sf +++ [NAY]
compile (OpB o v1 v2) sf = compileOp o v1 v2 sf where
  partial 
  compileOp : BinOp a b c -> Expr G a -> Expr G b -> 
                StackFrame G n -> getProg s (getEff c)
  compileOp Add e1 e2 sf = compile e1 sf +++ compile e2 sf +++ [ADD]
  compileOp Sub e1 e2 sf = compile e1 sf +++ compile e2 sf +++ [SUB]
  compileOp Mul e1 e2 sf = compile e1 sf +++ compile e2 sf +++ [MUL]
  compileOp Div e1 e2 sf = compile e1 sf +++ compile e2 sf +++ [DIV]
  compileOp Eql e1 e2 sf = compile e1 sf +++ compile e2 sf +++ [EQL]
  compileOp Lt  e1 e2 sf = compile e1 sf +++ compile e2 sf +++ [LTH]
compile (If b tb fb) {t} sf with (t)
  | TipUnit = []
  | TipBool = compile tb sf +++ compile fb sf +++ compile b  sf +++ [IF]
  | TipInt  = compile tb sf +++ compile fb sf +++ compile b  sf +++ [IF]
compile (App (Lam b) e) sf = compile b (e ::: sf)
compile (Var stop) {G = x :: xs} (e ::: sf) = compile e sf
compile (Var (pop k)) (e ::: sf) = compile (Var k) sf
\end{alltt}
\label{fig:compile_function}
\caption{Our compile function. It is partial as it does not cover all constructors of \texttt{Expr}. \texttt{compileOp} is partial since it calls \texttt{compile}.}
\end{figure}
\subsection{Run}
% What does all this mean when we run our program
