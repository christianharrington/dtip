%!TEX root = ../main.tex
\subsection{Programs}
\label{sec:program}
As previously mentioned, a stack machine executes a program as a list of instructions. The instruction set of our machine can be seen in Figure~\ref{fig:inst_set}, along with the stack effect of each instruction.

\begin{figure}
\centering{
\begin{tabularx}{\textwidth}{ | X X X | }
  \hline
  \textbf{Instruction} & \textbf{Stack before} & \textbf{Stack after} \\ \hline
  \texttt{PUSH} $i$ & $s$ & $s$, $i$ \\ 
  \texttt{ADD} & $s$, $i_{1}$, $i_{2}$ & $s$, $(i_{1}+i_{2})$ \\
  \texttt{SUB} & $s$, $i_{1}$, $i_{2}$ & $s$, $(i_{1}+i_{2})$ \\
  \texttt{MUL} & $s$, $i_{1}$, $i_{2}$ & $s$, $(i_{1}+i_{2})$ \\
  \texttt{DIV} & $s$, $i_{1}$, $i_{2}$ & $s$, $(i_{1}+i_{2})$ \\
  \texttt{EQL} & $s$, $i_{1}$, $i_{2}$ & $s$, $(i_{1}+i_{2})$ \\
  \texttt{LTH} & $s$, $i_{1}$, $i_{2}$ & $s$, $(i_{1}+i_{2})$ \\
  \texttt{NAY} & $s$, $i_{1}$, $i_{2}$ & $s$, $(i_{1}+i_{2})$ \\
  \texttt{IF}  & $s$, $b$, $e_{1}$, $e_{2}$ & $s$, (\texttt{if} $b$ \texttt{then} $e_{1}$ \texttt{else} $e_{2}$) \\ \hline
\end{tabularx}
}
\caption{Instruction set where $s$ represents the part of the stack untouched by the instruction. $i_n$ denotes integer values, while $b$ denotes Boolean values.}
\label{fig:inst_set}
\end{figure}

Looking at Figure~\ref{fig:inst_set}, the problem of stack underflow becomes very apparent. Each instruction requires a specific number of elements on the stack, and since this is not encoded in the instruction, these stack requirements must be enforced at run time rather than at compile time. Using dependent types, we can ensure these requirements are met statically. This can be done by encoding the required stack size for an instruction in the type of the instruction. Just indexing instructions by their required stack size, however, is not sufficient. Since we want to be able to create a sequence of instructions, each instruction needs to know how many values the previous instruction left on the stack. This means an instruction must be indexed by the complete stack effect. 

\begin{figure}
\centering{
\begin{tabularx}{\textwidth}{ | X X X | }
  \hline
  \textbf{Instruction} & \textbf{Minimum stack size} & \textbf{Values left on stack} \\ \hline
  \texttt{PUSH} $i$ & $0$ & $1$ \\ 
  \texttt{ADD} & $2$ & $1$ \\
  \texttt{SUB} & $2$ & $1$ \\
  \texttt{MUL} & $2$ & $1$ \\
  \texttt{DIV} & $2$ & $1$ \\
  \texttt{EQL} & $2$ & $1$ \\
  \texttt{LTH} & $2$ & $1$ \\
  \texttt{NAY} & $2$ & $1$ \\
  \texttt{IF}  & $3$ & $1$ \\ \hline

\end{tabularx}
}
\caption{Instruction set with stack effects.}
\label{fig:inst_set_with_effect}
\end{figure}

\begin{figure}
\begin{alltt}
data Inst : Nat \(\rightarrow\) Nat \(\rightarrow\) Type where
  PUSH : Int \(\rightarrow\) Inst          s    (S s)
  ADD  :        Inst    (S (S s))  (S s)
  SUB  :        Inst    (S (S s))  (S s)
  MUL  :        Inst    (S (S s))  (S s)
  DIV  :        Inst    (S (S s))  (S s)
  EQL  :        Inst    (S (S s))  (S s)
  LTH  :        Inst    (S (S s))  (S s)
  NAY  :        Inst       (S s)   (S s)
  IF   :        Inst (S (S (S s))) (S s)
\end{alltt}
\caption{Idris implementation of instructions}
\label{fig:idris_inst_set}
\end{figure}

Since the stack effect of an instruction represents a change in stack size, the indices must be based on the prior size of the stack, which is an implicit argument. This is expressed in Figure~\ref{fig:idris_inst_set}. For example the type of the \texttt{PUSH i} instruction is \texttt{Inst s (S s)} rather than \texttt{Inst 0 1}, which one might expect from Figure~\ref{fig:inst_set_with_effect}. 

\subsubsection{Program Representation}
To avoid a stack underflow error, the type of a program must encode its stack effect. This can be done in the same way as with instructions. Thus the data type for a program becomes:
\begin{alltt}
data Prog : Nat \(\rightarrow\) Nat \(\rightarrow\) Type where
\end{alltt}
The stack effect of a program is directly inferable from the instructions it contains. To do this we use a structure similar to a list, as seen in Figure~\ref{fig:idris_impl_of_prg}. An empty program (\texttt{Nil}) is a program that does not change the stack. Adding an instruction to an existing program (\texttt{::}) will construct a new program, with a stack effect modified by the instruction. Since instruction indices are universally quantified, any instruction will fit on any program. This is potentially dangerous, as it is possible to construct programs that will cause a stack underflow error if run with an insufficient initial stack. As a result, the stack machine must require an initial stack of the correct size before running a program, which will be further discussed in Section~\ref{sec:running_a_program}.

\begin{figure}
\begin{alltt}
data Prog : Nat \(\rightarrow\) Nat \(\rightarrow\) Type where
  Nil  : Prog s s
  (::) : Inst s s\('\) \(\rightarrow\) Prog s\('\) s\('\)\('\) \(\rightarrow\) Prog s s\('\)\('\)
\end{alltt}
\caption{Idris representation of well-stacked programs}
\label{fig:idris_impl_of_prg}
\end{figure}

In addition to constructing programs from scratch, it can also be useful to append one program to another. Due to the similarities between our program representation and lists, we use a similar name and notation to the \texttt{++} function found in many functional programming languages. Although it is defined as \texttt{+++} in Idris, we will refer to it as \texttt{append}.

\begin{alltt}
(+++) : Prog s s\('\) -> Prog s\('\) s\('\)\('\) -> Prog s s\('\)\('\)
(+++) Nil p2       = p2
(+++) (i :: p1) p2 = i :: (p1 +++ p2)
\end{alltt}

The type of \texttt{append} specifies that given two programs with stack effects $(s, s')$ and $(s', s'')$, we can construct a program with stack effect $(s, s'')$. Because the stack effects are implicit arguments to the \texttt{append} function, any two programs can be concatenated, if the Idris type checker can infer the arguments.
