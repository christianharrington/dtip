%!TEX root = ../main.tex
\subsection{Omissions}
\label{sec:omissions}
As previously mentioned, our \texttt{compile} function, and thereby our stack machine, does not cover our entire programming language. Specifically sum types, product types, and fixed points are not handled. 

\paragraph{Sum and Product Types}
When defining the instructions corresponding to sums and products, we run into a problem. This is because the stack effect of these terms depend on their contents. All other instructions always have the same effect. We need some way to represent instructions with variable stack effects.

\paragraph{Fixed points}
Compiling a fixed point for a stack machine that is not well-stacked is fairly straightforward. If treated like a function call, we could simply have a \texttt{CALL} instruction pointing to the start of the function with a conditional. With our well-stacked machine this is not as simple, since a \texttt{CALL} instruction in our representation would always have the same stack effect. Once again, we are in need of some way of representing instructions with varying stack effects.

\subsubsection{Possible Solution}
Both of the above issues could be solved by having an instruction where the effect is based on some parameter. We can use our previously defined \texttt{Eff} type for this. Consider the two functions, \texttt{stackReq} and \texttt{stackProduce}, in Figure~\ref{fig:stack_effect_functions}. These functions produce a natural number based on an \texttt{Eff} and a natural number. These can be used to construct a variable instruction based on an \texttt{Eff} also seen in Figure~\ref{fig:stack_effect_functions}. 

\begin{figure}
\begin{alltt}
stackProduce : Eff \(\to\) Nat \(\to\) Nat
stackProduce Flat n = n
stackProduce (Inc Z) n = S n
stackProduce (Inc m) n = S (m+n)
stackProduce (Dec _) n = n

stackReq : Eff \(\to\) Nat \(\to\) Nat
stackReq Flat n = n
stackReq (Inc x) n = n
stackReq (Dec Z) n = S n
stackReq (Dec m) n = S (m+n)
\end{alltt}
\caption{Functions creating natural numbers from \texttt{Eff}s used for indexing instructions.}
\label{fig:stack_effect_functions}
\end{figure}

We could use these functions to create a \texttt{CALL} instruction with a stack effect based on an \texttt{Eff}. This \texttt{CALL} instruction could be used to implement function calls, and thereby fixed points from our well-typed representation. To do this we would need an additional layer of abstraction to our current program representation. This could be a vector containing multiple instances of our current program representation, where each element in the vector can be thought of as a function. For this to work we need to encode more information into the \texttt{CALL} instruction. Firstly, we need a program environment which contains the programs this instruction can \emph{call}. Secondly, we need an index into this environment to specify what function to call.

In Section~\ref{sec:a-well-typed-expression-language} we describe how variables are represented in our well-typed representation by using a predicate. We can use the same approach for function calls, by having our \texttt{CALL} instruction parameterized by a membership predicate on its enclosing environment. This can be used as a intrinsic proof that a program at a given position in the environment has a specific stack effect. 

\begin{figure}
\begin{alltt}
data HasEffect : Vect n Eff \(\to\) Eff \(\to\) Type where
    stop : HasEffect (e :: E) e
    pop  : HasEffect E e \(\to\) HasEffect (f :: E) e

data Inst : Nat \(\to\) Nat \(\to\) Type where
    \vdots    
    CALL : (e : Eff) \(\to\) HasEffect E e \(\to\) 
           Inst (stackReq e s) (stackProduce e s)
\end{alltt}
\caption{\texttt{HasEffect} membership predicate and \texttt{CALL} instruction with variable indices based on an \texttt{Eff}.}
\label{fig:call}
\end{figure}

Figure~\ref{fig:call} shows this membership predicate (\texttt{HasEffect}) and an instruction used for function calls. For constructing a \texttt{CALL} instruction we need an \texttt{Eff} and our membership predicate. For the sake of simplicity we will henceforth use a numerical representation for the index (i.e. \texttt{stop} is \texttt{0}, \texttt{pop stop} is \texttt{1} etc.). The intuition behind \texttt{HasEffect} is analogous to the one behind \texttt{HasType} in the well-typed representation.

\begin{figure}
\begin{alltt}
p 0: [PUSH 2, CALL (Inc 0) 1, ADD]
     ::
p 1: [PUSH 1, CALL (Inc 0) 2, ADD]
     ::
p 2: Nil
     ::
p 3: [PUSH 4]
     ::
p 4: Nil
\end{alltt}
\caption{Possible program structure with functions. p 0--4 are the programs in the program vector.}
\label{fig:program_structure}
\end{figure}

In Figure~\ref{fig:program_structure}, \texttt{p 0} is the main program. \texttt{CALL (Inc 0) 1} indicates a call to the next program in the program vector with stack effect \texttt{(s,~S~s)}, \texttt{p 1}. \texttt{p 1} contains the instruction \texttt{CALL (Inc 0) 2}, which indicates another call, this time to the program two elements further in the program vector. Note that this is a call to \texttt{p 3} as the parameter of \texttt{CALL} is relative to the current program, and not a global index in the program vector. 

Compiling a fixed point term from Figure~\ref{fig:idris-def-expr-lang} (\texttt{Fix e}) would mean constructing a \texttt{CALL} instruction based on the type of the term, after which the term \texttt{e} is compiled and added to both the program vector and the environment. Since the type of the term being compiled determines the stack effect of the resulting program, the effect of the \texttt{CALL} instruction and the compiled program will be identical. When running the program this correspondence between effects will allow us to run the function in place of the call, as their stack effects match.

Note that since we can only call programs further down the program vector, mutual recursion is not possible. Regular recursion is available through \texttt{CALL~e~0}.

Interestingly, the above way of handling function calls also be used for product and sum types. For example, treating each part of a product type as a function, and \texttt{Fst} and \texttt{Snd} as function calls would be one way to implement this.

It is worth noting that this has not been fully implemented due to time constraints. The shown implementations of \texttt{stackReq}, \texttt{stackProduce}, \texttt{HasEffect}, and \texttt{CALL} are all accepted by the Idris type checker, but there might be complications when completing the implementation requiring alterations to these definitions. Furthermore, fully implementing these changes would involve revamping many parts of our stack machine implementation to handle a vector of programs rather than just a program.
