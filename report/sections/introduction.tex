%!TEX root = ../main.tex
\section{Introduction}
\label{sec:introduction}
% - Begrundelse
% - Problemets karakter: Løst tidligere
%   - Tilgang er anderledes
% - Udfordring: Dependendt types
% - Vores forudsætninger
% - Opbygning
%   - Raw STLC i Idris
%   - Well-typed expression language
%   - Type Checking
%   - Stack machine

This report describes the design and implementation of a simple functional programming language, using dependent types. This programming language is based on the simply typed lambda calculus with several common extensions, as described in \emph{Types and Programming Languages} by Benjamin C. Pierce\,\cite{Pierce:TypeSystems}. The programming language is not interesting in and of itself. Languages like this have been implemented thousands of times in as many programming languages. Instead, we will focus on how dependent types can be used in the abstract representation of the language to increase our confidence that the implementation is correct. We will not implement a concrete syntax nor lexer/parser for the language. Our approach is inspired by the ``well-typed interpreter'' presented by Augustsson and Carlsson\,\cite{Augustsson99anexercise}. 

The abstract representation, along with all other components, was written in Idris\,\cite{Idris}, a dependently typed general purpose programming language developed primarily by Edwin Brady of the University of St Andrews.  

Programming with dependent types is not yet widespread. Therefore this report has been written so readers with no previous knowledge of dependent types should be able to understand it. Besides general programming experience, the only prerequisite for understanding this paper is proficiency with a functional programming language, preferably Haskell, since the Idris syntax is very similar. All code examples in this paper are written in Idris.

\subsection{Structure}
In Section~\ref{sec:background}, we begin by giving a quick overview of the simply typed lambda calculus. This is followed by a brief introduction to dependent types. Then, in Section~\ref{sec:the-simply-typed-lambda-calculus}, a na\"{i}ve representation of our programming language is presented. This will serve as the base for our well-typed representation in Section~\ref{sec:a-well-typed-expression-language}. In Section~\ref{sec:type-checking}, a type-checking system is introduced, which lets us transform expressions from the na\"{i}ve to the well-typed representation. As the final link in the chain, a compiler and stack machine are described in Section~\ref{sec:stack-machine}. This lets us compile our well-typed representation to a ``well-stacked'' program, in which stack underflows are statically prevented. In Section~\ref{sec:partiality}, the consequences of introducing general recursion into the programming language are discussed. Because a primary goal of this project was to investigate dependently typed programming from a practical perspective, Section \ref{sec:practical-reflections} documents our experiences with the hands-on aspects of Idris programming. Section~\ref{sec:related-work} lists related work, and Section~\ref{sec:conclusion} offers concluding thoughts.
