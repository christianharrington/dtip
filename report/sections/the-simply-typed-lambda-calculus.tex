%!TEX root = ../main.tex
\section{A ``Raw'' Representation}
\label{sec:the-simply-typed-lambda-calculus}
% - Idris definitions
% - Eksempler 
% 	- Virkende
% 	- Uvirkende (Thomas tvang mig til at skrive dette dårlige ord)
% - Extensions

Our first task is to define which features our programming language will include. After this, we will define a ``raw'' representation of the language. When we say ``raw'', we mean that we have not done any type checking yet, and will not use dependent types. This representation can be thought of as the output of a hypothetical parser. 

As we have mentioned earlier, our programming language will be based on the simply typed lambda calculus, with several extensions. These extensions have not been chosen based on any specific design principles; instead, they are simply there to make the language more expressive and interesting. The extensions are:
\begin{itemize}
\item Base types\,\cite[pp. 117]{Pierce:TypeSystems}: Boolean, integer, and unit values.
\item Operators: Select binary and unary primitive operators.
\item Product types\,\cite[pp. 126]{Pierce:TypeSystems}: For working with pairs of terms.
\item Sum types\,\cite[pp. 132]{Pierce:TypeSystems}: For handling terms that can have several fixed types.
\item Conditionals: Of the form \texttt{if $e_{1}$ then $e_{2}$ else $e_{3}$}.
\item General recursive functions\,\cite[pp. 142]{Pierce:TypeSystems}: For constructing general recursive functions.
\end{itemize}

It should be noted that the addition of the last extension, general recursion, has consequences. Notably, our language is no longer strongly normalizing, i.e. we do not know if evaluation of a term will halt in finite number of steps. For a full discussion of the consequences of this inclusion, see Section~\ref{sec:partiality}.

Now that we have defined our programming language, we need a way to represent it abstractly. In Figure~\ref{fig:rstlc}, the raw representation is given. A term in this representation forms an abstract syntax tree. The \verb+--+ characters denote a comment, which in this case list which extension a given term is enabling.

\begin{center}
\begin{figure}
\begin{alltt}
data Expr : Type where
  Lam  : Expr \(\to\) Expr                 -- STLC: Function introduction
  Var  : Nat \(\to\) Expr                  -- STLC
  App  : Expr \(\to\) Expr \(\to\) Expr         -- STLC: Function elimination
  Val  : Int \(\to\) Expr                  -- Base type: Int introduction
  Boo  : Bool \(\to\) Expr                 -- Base type: Bool introduction
  Unit : Expr                         -- Base type: Unit introduction
  OpB  : BinOp \(\to\) Expr                -- Binary operation
  OpU  : UnOp \(\to\) Expr                 -- Unary operation
  Pair : Expr \(\to\) Expr \(\to\) Expr         -- Product type: Pair introduction
  Fst  : Expr \(\to\) Expr                 -- Product type: Pair elimination
  Snd  : Expr \(\to\) Expr                 -- Product type: Pair elimination
  InL  : Expr \(\to\) Expr                 -- Sum type: Sum introduction
  InR  : Expr \(\to\) Expr                 -- Sum type: Sum introduction
  Case : Expr \(\to\) Expr \(\to\) Expr \(\to\) Expr -- Sum type: Sum elimination
  If   : Expr \(\to\) Expr \(\to\) Expr \(\to\) Expr -- Conditional: Bool elimination
  Fix  : Expr \(\to\) Expr                 -- General recursion
\end{alltt}
\caption{The raw representation of our programming language in Idris. The comments specify which language feature they help implement, and where applicable, whether it introduces or eliminates an element.}
\label{fig:rstlc}
\end{figure}
\end{center}

Besides these terms, there are data types for representing various binary and unary operations, such as \texttt{Plus}, \texttt{Minus} and so on. We use de Bruijn indices to represent variables, so we do not have to concern ourselves with naming. For a deeper discussion of our use of de Bruijn indices, see Section~\ref{sec:making-the-representation-well-typed}. A very simple function that adds some number to a fixed value can be represented with \texttt{Lam (OpB (Plus (Var 1) (Val 5)))}. The problem with this representation of our programming language is that it is too permissive. It lets us write completely nonsensical expressions, such as \texttt{Lam (OpB (Plus (Pair (Val 2) (Boo True)) Unit))}. How is one supposed to add the pair \texttt{(2, True)} to unit? Looking past this obviously incorrect example, lets go back to the first example. We have no idea if there is a variable at index 1. If it does exist, it might be a Boolean or some other type that we cannot add to 5. It is clear that with this representation, we will need numerous run time checks to make sure everything is in order. 

We could run an expression in this representation through a type checker to merely find out if it was well-typed or not, but instead we will show how we can use dependent types to define a well-typed representation, where it is impossible to build ill-typed or ill-scoped expressions. This is an improvement, as we will be able to statically find errors in the handling of the representation. For example, when writing an optimization function for the well-typed expression language, we can not accidentally produce an ill-typed expression. Specifically, our well-typed representation will guarantee:

\begin{itemize}
\item All terms are well-typed.
\item Only variables in scope can be accessed.
\item Only lambda and case terms can introduce new variables.
\end{itemize}

In the next section, we will specify this well-typed representation, and in Section~\ref{sec:type-checking} we will explore how we can transform this raw representation to the well-typed representation using a type checker.
